\section{Hybrid chat application}


Thiết kế này áp dụng nguyên tắc phân tách trách nhiệm (Separation of Concerns) rõ ràng giữa hai thành phần chính:
\begin{itemize}
    \item Web Server (start\_app.py): Đóng vai trò là "người điều phối". Server chịu trách nhiệm quản lý việc xác thực (authentication) người dùng, phân quyền (authorization), tạo channel, quản lý thành viên (ai được tham gia) và lưu trữ mật khẩu (dưới dạng hash).
    \item P2P Client (start\_p2p.py): Chịu trách nhiệm xử lý toàn bộ logic nhắn tin (messaging). Sau khi được Web Server "giới thiệu" và cung cấp danh sách thành viên, P2P Client sẽ thực hiện broadcast tin nhắn trực tiếp đến các thành viên khác trong channel mà không cần thông qua server.
\end{itemize}



\subsection{Chatting between 2 peer}

% \subsection{Peer-to-Peer Paradigm}
Hệ thống hỗ trợ hai chức năng chính trong mô hình P2P:
\begin{itemize}
    \item Giao tiếp trực tiếp 1--1 giữa các peer.
    \item Broadcast tin nhắn đến tất cả peer trong mạng.
\end{itemize}
Luồng hoạt động chính: 
\begin{itemize}

    \item Gửi thông tin địa chỉ — \textbf{@app.route('/submit-info', methods=['POST'])}:  
    Người dùng gửi thông tin địa chỉ \textit{(chatting IP, port, local-port)} trước khi tham gia mạng P2P.
    \begin{enumerate}
        \item Người dùng nhập thông tin địa chỉ thông qua form trên giao diện.
        \item Hệ thống kiểm tra tính hợp lệ của địa chỉ và thực hiện xác thực người dùng.
        \item Sau khi xác thực thành công, hệ thống lấy \textit{username} và lưu thông tin địa chỉ tương ứng vào bảng \texttt{account\_to\_address}.
    \end{enumerate}

    \item Xem danh sách các người dùng — \textbf{@app.route('/get-list', methods=['GET'])}:  
    Hỗ trợ người dùng xem danh sách peer đang online trong mạng P2P.
    \begin{enumerate}
        \item Hệ thống xác thực người dùng và kiểm tra xem người dùng đã khai báo địa chỉ hay chưa.
        \item Duyệt toàn bộ bảng \texttt{account\_to\_address} để xây dựng danh sách peer đang hoạt động.
        \item Với mỗi peer, hệ thống sinh một form HTML cho phép kết nối trực tiếp 1–1 thông qua endpoint \texttt{/connect-peer}.
        \item Đồng thời tạo form broadcast, chứa danh sách toàn bộ peer còn lại để gửi tin nhắn diện rộng.
    \end{enumerate}

\end{itemize}

\subsubsection{Giao tiếp trực tiếp 1-1 giữa peer}

Chức năng giao tiếp trực tiếp giữa các peer được hiện thực theo cơ chế trao đổi tin nhắn giữa hai peer không phụ thuộc vào bất kỳ máy chủ trung gian nào; thay vào đó, hai ứng dụng liên kết trực tiếp với nhau thông qua kết nối TCP socket.

Luồng hoạt động chi tiết như sau:

\begin{itemize}

    \item Kết nối vào đoạn chat — \textbf{@app.route("/connect-peer", methods=["POST"])}:  
    Người dùng muốn thiết lập kết nối vào đoạn chat với một người dùng khác.
    \begin{enumerate}
        \item Hệ thống lưu thông tin địa chỉ (\textit{server\_ip, server\_port}) của người dùng hiện tại.
        \item Sau đó, lấy thông tin địa chỉ của peer mà người dùng muốn kết nối.
        \item Tiếp theo thực hiện chuyển hướng sang trang \texttt{/chat} và kèm theo tham số địa chỉ của peer cần kết nối.
        \item Tại \textbf{@app.route("/chat", methods=["GET"])}, hệ thống kiểm tra \texttt{chat\_history} để lấy lại nội dung hội thoại đã lưu tương ứng.
    \end{enumerate}

    \item Gửi tin nhắn — \textbf{@app.route("/chat", methods=["POST"])}:  
    Người dùng gửi một tin nhắn mới trong giao diện chat.
    \begin{enumerate}
        \item Hệ thống lấy địa chỉ peer qua \texttt{headers["query"]} và nội dung tin nhắn từ \texttt{body["message"]}.
        \item Sau đó đưa tin nhắn này vào \texttt{send\_queue} dưới dạng một tuple \texttt{(peer\_ip, peer\_port, message\_to\_send)}.
        \item \textit{Sender thread} liên tục đọc tin nhắn từ hàng đợi (\texttt{send\_queue}).
        \item Sender thread gọi \texttt{send\_message()}, tạo timestamp và đóng gói tin nhắn theo đúng định dạng chuẩn.
        \item Hàm \texttt{send\_message()} mở một socket TCP tạm thời, kết nối đến địa chỉ của peer đích và chuyển tiếp gói tin.
        \item Sau khi gửi thành công hoặc xảy ra lỗi, socket được đóng và cập nhật \texttt{chat\_history} cho peer tương ứng.
    \end{enumerate}

    \item Nhận tin nhắn — mỗi peer duy trì một \textit{server\_thread} thực thi hàm \texttt{start\_server()} mở một socket TCP tại địa chỉ \texttt{0.0.0.0:port} và lắng nghe mọi kết nối đến. Khi có một peer khác gửi tin đến:
    \begin{enumerate}
        \item Đầu tiên, server gọi \texttt{accept()} và tạo một handler thread mới để xử lý kết nối đó.
        \item Tiếp theo đọc thông điệp từ socket, giải mã chuỗi tin nhắn theo đúng định dạng chuẩn và phân tách ba thành phần: sender, timestamp và nội dung.
        \item Hệ thống hiển thị thông tin tin nhắn lên giao diện console và cập nhật nội dung vào \texttt{chat\_history}.  
        \item Và khi xử lý xong, kết nối socket được đóng lại.
    \end{enumerate}

\end{itemize}

\subsubsection{Cơ chế broadcast}

Bên cạnh chức năng giao tiếp 1-1 giữa hai peer, hệ thống còn hỗ trợ cơ chế gửi tin nhắn theo mô hình broadcast, tức là một peer có thể gửi một thông điệp đồng thời đến toàn bộ các peer khác trong mạng. Broadcast là hành động gửi một gói tin duy nhất từ peer nguồn đến tập tất cả các peer được lưu trong danh sách \texttt{peer\_list}.

Luồng hoạt động như sau:

\begin{itemize}

    \item Xác định thông tin địa chỉ — \textbf{@app.route("/broadcast0", methods=["POST"])}:  
    Người dùng muốn gửi tin nhắn đến mọi người khác:
    \begin{enumerate}
        \item Hệ thống lưu thông tin địa chỉ (server\_ip, server\_port) của người dùng hiện tại.
        \item Sau đó, lấy danh sách thông tin địa chỉ của các peer trong peer-list.
        \item Tiếp theo hệ thống xác thực người dùng và thực hiện chuyển hướng sang trang \texttt{/broadcast} để hiển thị giao diện tương ứng.
    \end{enumerate}

    \item Gửi tin nhắn — \textbf{@app.route("/broadcast", methods=["POST"])}:  
    Người dùng gửi một tin nhắn broadcast từ giao diện:
    \begin{enumerate}
        \item Đầu tiên, hệ thống lấy nội dung tin nhắn và tự động thêm tiền tố ``[Broadcast]'' để đánh dấu đây là dạng tin broadcast.
        \item Hệ thống duyệt qua toàn bộ \texttt{peer\_list}, và với mỗi peer sẽ đưa vào \texttt{send\_queue} một yêu cầu gửi tin: (peer\_ip, peer\_port, broadcast\_message).
        \item Tiếp theo queue này sẽ được xử lý bởi sender thread.
    \end{enumerate}

    \item Nhận tin nhắn broadcast: Vì mỗi peer duy trì một \texttt{server\_thread} mở một socket để lắng nghe mọi tin được gửi đến nên phần này hoạt động giống như P2P 1-1.
\end{itemize}










\subsection{Channel management}


Tính năng Channel Chat được thiết kế để cho phép nhiều người dùng tham gia vào một phòng chat chung , song song với tính năng chat 1-1 hiện có. Mỗi channel được xác định duy nhất bằng tên (channel name) và được bảo vệ bằng mật khẩu để kiểm soát truy cập.






 Luồng hoạt động chi tiết như sau:

\begin{itemize}
    % \item 

    \item Tạo Channel mới - \textbf{@app.route("/create-channel", methods=["POST"])}: Người dùng khởi tạo một phòng chat mới.
    \begin{enumerate}
        \item  Đầu tiên, User cung cấp channel\_name và password qua form.
        \item Tiếp theo Server kiểm tra channel\_name có tồn tại trong hệ thống chưa. Nếu đã tồn tại thì báo lỗi cho người dùng.
        \item Nếu chưa thì tuần tự Server thực hiện hash(password) để lưu trữ an toàn, sau đó tạo một mục mới trong cấu trúc dữ liệu channels và cuối cùng thêm địa chỉ của người dùng vào danh sách thành viên đầu tiên.
    \end{enumerate}


   \item Xem danh sách và Tham gia Channel - \textbf{@app.route("/channel", methods=["GET"])} và \textbf{@app.route("/join-channel", methods=["POST"])}: Cho phép người dùng thấy các channel có sẵn và tham gia chúng. Với xem luồng danh sách thì khi truy cập channel, server hiển thị 2 danh sách bao gồm joined channels - Các channel mà người dùng đã là thành viên; và available channels - Các channel khác mà người dùng có thể tham gia. Còn đối với luồng tham gia:
   \begin{enumerate}
       \item User chọn một channel từ "Available channels" và nhập password.
       \item Server lấy password người dùng nhập, thực hiện hash và so sánh với stored\_password\_hash của channel đó.
       \item Nếu hash(password) == stored\_password\_hash, thêm địa chỉ của user (user\_address) vào address\_list của channel kèm thông báo tham gia thành công. Nếu sai, báo lỗi mật khẩu không chính xác.
   \end{enumerate}



   \item Kết nối vào Channel Chat (Handoff to P2P) - \textbf{@app.route("/connect-channel", methods=["POST"])}: Khởi tạo phiên chat P2P sau khi người dùng đã được xác thực là thành viên.
   \begin{enumerate}
       \item User nhấp vào "Connect" trên một channel đã tham gia.
       \item Web Server lấy address\_list (danh sách địa chỉ IP/Port) của tất cả thành viên hiện tại trong channel đó.
       \item Server thực hiện một POST redirect (chuyển hướng POST) sang ứng dụng P2P Client (start\_p2p.py), mang theo address\_list này.
       \item P2P Client nhận được danh sách, lưu vào channel\_members[channel\_name] và thực hiện broadcast một thông báo "join" đến tất cả các thành viên trong danh sách.
   \end{enumerate}



   \item Chat trong Channel: Luồng gửi và nhận tin nhắn P2P. Đây là luồng gửi tin nhắn:
   \begin{enumerate}
       \item  User gõ tin nhắn trong giao diện (UI) của channel.
       \item P2P App (của người gửi) xác định channel hiện tại (channel\_name).
       \item App lấy danh sách thành viên từ channel\_members[channel\_name].
       \item App thực hiện broadcast (gửi đồng thời) tin nhắn đến tất cả địa chỉ trong danh sách đó.
   \end{enumerate}
   Còn đây là luồng nhận tin nhắn:
   \begin{enumerate}
       \item P2P Client (của người nhận) lắng nghe và nhận được một tin nhắn.
       \item Client kiểm tra định dạng tin nhắn.
       \item Client phân tích (parse) tin nhắn để lấy sender\_address, timestamp, channel\_name, và message.
       \item Client xác định đây là tin nhắn của channel\_name và hiển thị nội dung message lên UI của channel tương ứng.
   \end{enumerate}
\end{itemize}



Để cụ thể hơn về định dạng tin nhắn được nhắc ở trên thì cần phân tích sâu hơn ở đặc điểm chính là để phân biệt tin nhắn. Để phân biệt tin nhắn 1-1 và tin nhắn channel, một định dạng prefix đặc biệt được sử dụng ở trong này. Client P2P sẽ dựa vào prefix này để xử lý tin nhắn đúng cách.

Định dạng này được định nghĩa như sau: [sender\_address] [timestamp] [Channel]: channel\_name  message. Tiền tố [Channel]channel\_name là dấu hiệu nhận biết (flag) cho P2P Client rằng đây là tin nhắn thuộc về một channel cụ thể.
